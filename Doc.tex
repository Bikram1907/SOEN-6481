%%%%%%%%%%%%  Generated using docx2latex.com  %%%%%%%%%%%%%%

%%%%%%%%%%%%  v2.0.0-beta  %%%%%%%%%%%%%%

\documentclass[12pt]{article}
\usepackage{amsmath}
\usepackage{latexsym}
\usepackage{amsfonts}
\usepackage[normalem]{ulem}
\usepackage{array}
\usepackage{amssymb}
\usepackage{graphicx}
\usepackage[backend=biber,
style=numeric,
sorting=none,
isbn=false,
doi=false,
url=false,
]{biblatex}\addbibresource{bibliography.bib}

\usepackage{subfig}
\usepackage{wrapfig}
\usepackage{wasysym}
\usepackage{enumitem}
\usepackage{adjustbox}
\usepackage{ragged2e}
\usepackage[svgnames,table]{xcolor}
\usepackage{tikz}
\usepackage{longtable}
\usepackage{changepage}
\usepackage{setspace}
\usepackage{hhline}
\usepackage{multicol}
\usepackage{tabto}
\usepackage{float}
\usepackage{multirow}
\usepackage{makecell}
\usepackage{fancyhdr}
\usepackage[toc,page]{appendix}
\usepackage[hidelinks]{hyperref}
\usetikzlibrary{shapes.symbols,shapes.geometric,shadows,arrows.meta}
\tikzset{>={Latex[width=1.5mm,length=2mm]}}
\usepackage{flowchart}\usepackage[paperheight=11.0in,paperwidth=8.5in,left=1.0in,right=1.0in,top=1.0in,bottom=1.0in,headheight=1in]{geometry}
\usepackage[utf8]{inputenc}
\usepackage[T1]{fontenc}
\TabPositions{0.5in,1.0in,1.5in,2.0in,2.5in,3.0in,3.5in,4.0in,4.5in,5.0in,5.5in,6.0in,}

\urlstyle{same}


 %%%%%%%%%%%%  Set Depths for Sections  %%%%%%%%%%%%%%

% 1) Section
% 1.1) SubSection
% 1.1.1) SubSubSection
% 1.1.1.1) Paragraph
% 1.1.1.1.1) Subparagraph


\setcounter{tocdepth}{5}
\setcounter{secnumdepth}{5}


 %%%%%%%%%%%%  Set Depths for Nested Lists created by \begin{enumerate}  %%%%%%%%%%%%%%


\setlistdepth{9}
\renewlist{enumerate}{enumerate}{9}
		\setlist[enumerate,1]{label=\arabic*)}
		\setlist[enumerate,2]{label=\alph*)}
		\setlist[enumerate,3]{label=(\roman*)}
		\setlist[enumerate,4]{label=(\arabic*)}
		\setlist[enumerate,5]{label=(\Alph*)}
		\setlist[enumerate,6]{label=(\Roman*)}
		\setlist[enumerate,7]{label=\arabic*}
		\setlist[enumerate,8]{label=\alph*}
		\setlist[enumerate,9]{label=\roman*}

\renewlist{itemize}{itemize}{9}
		\setlist[itemize]{label=$\cdot$}
		\setlist[itemize,1]{label=\textbullet}
		\setlist[itemize,2]{label=$\circ$}
		\setlist[itemize,3]{label=$\ast$}
		\setlist[itemize,4]{label=$\dagger$}
		\setlist[itemize,5]{label=$\triangleright$}
		\setlist[itemize,6]{label=$\bigstar$}
		\setlist[itemize,7]{label=$\blacklozenge$}
		\setlist[itemize,8]{label=$\prime$}

\setlength{\topsep}{0pt}\setlength{\parskip}{8.04pt}
\setlength{\parindent}{0pt}

 %%%%%%%%%%%%  This sets linespacing (verticle gap between Lines) Default=1 %%%%%%%%%%%%%%


\renewcommand{\arraystretch}{1.3}


%%%%%%%%%%%%%%%%%%%% Document code starts here %%%%%%%%%%%%%%%%%%%%



\begin{document}
{\fontsize{14pt}{16.8pt}\selectfont \textbf{Problem 1. [20 Marks]}\par}\par

\begin{Center}
\textbf{pi (}{\fontsize{14pt}{16.8pt}\selectfont \textbf{\textcolor[HTML]{323232}{$ \pi $ )}}\par}
\end{Center}\par

\begin{justify}
\textbf{Definition: }pi is the ratio of the circumference of any circle to the diameter of that circle. Regardless of the size of the circle, this ratio will always equal pi. It is more commonly denoted by Greek letter ‘$ \pi $ ’ and is approximately equal to 3.14 in decimal form [1].
\end{justify}\par

\begin{justify}
\textbf{Origin: }The importance of pi has been recognized for at least 4,000 years. \textit{A History of Pi} notes that by 2000 B.C., "the Babylonians and the Egyptians (at least) were aware of the existence and significance of the constant $ \pi $ ," recognizing that every circle has the same ratio of circumference to diameter. The Greek letter $ \pi $  was first used for this purpose by William Jones in 1706, probably as an abbreviation of periphery, and became standard mathematical notation roughly 30 years later. Being an irrational number, with never ending digits after the decimal, by the start of 20\textsuperscript{th} century about 500 digits were known and with the help of modern computation advances we know more than the first six billion digits of pi [1].
\end{justify}\par

\begin{justify}
\textbf{Uniqueness: }Of all the irrational numbers, pi is one of the most popular and recognized mathematical constant. Its popularity can be assumed by the fact that mathematicians around the world celebrate March 14 (3/14) as Pi day. Some characteristics of pi which makes it special are:
\end{justify}\par

\begin{itemize}
	\item We can never truly calculate the area or the circumference of a circle because we can never truly know the value of pi.\par

	\item If we round the number pi to just 9 digits after the decimal and use it to calculate earth’s circumference, the results would be amazingly accurate. For every 25,000 miles, the number pi will only err to 1/4th of an inch. [2]\par

	\item $ \pi $  is also \textbf{transcendental}. That means that it is not the solution of any non-constant polynomial with rational coefficients. This is why we can't square the circle.\par

	\item There is even a connection between pi and gravity. Square root of the gravitational constant (9.8 m/s$ \string^ $ 2) is equal to 3.1305. [3] 
\end{itemize}\par

\begin{justify}
\textbf{Uses: }In mathematics, pi is almost used almost everywhere where a circle exists. We can find the use of pi in Geometry, trigonometry, vector calculus, Gaussian Integrals, Topology etc. [4].
\end{justify}\par

\begin{justify}
Apart from its vast usefulness in mathematics, Pi also appears in the physics that describes waves, such as ripples of light and sound. It even enters into the equation that defines how precisely we can know the state of the universe, known as Heisenberg's uncertainty principle [5].
\end{justify}\par

\begin{justify}
Pi even emerges in the shapes of rivers. A river's windiness is determined by its "meandering ratio," or the ratio of the river's actual length to the distance from its source to its mouth as the crow flies. Rivers that flow straight from source to mouth have small meandering ratios, while ones that lollygag along the way have high ones. Turns out, the average meandering ratio of rivers approaches pi. [5]
\end{justify}\par


\printbibliography
\end{document}