\documentclass[12pt]{article}
\usepackage{amsmath}
\usepackage{latexsym}
\usepackage{amsfonts}
\usepackage[normalem]{ulem}
\usepackage{array}
\usepackage{amssymb}
\usepackage{graphicx}
\usepackage[backend=biber,
style=numeric,
sorting=none,
isbn=false,
doi=false,
url=false,
]{biblatex}\addbibresource{bibliography.bib}

\usepackage{subfig}
\usepackage{wrapfig}
\usepackage{wasysym}
\usepackage{enumitem}
\usepackage{adjustbox}
\usepackage{ragged2e}
\usepackage[svgnames,table]{xcolor}
\usepackage{tikz}
\usepackage{longtable}
\usepackage{changepage}
\usepackage{setspace}
\usepackage{hhline}
\usepackage{multicol}
\usepackage{tabto}
\usepackage{float}
\usepackage{multirow}
\usepackage{makecell}
\usepackage{fancyhdr}
\usepackage[toc,page]{appendix}
\usepackage[hidelinks]{hyperref}
\usetikzlibrary{shapes.symbols,shapes.geometric,shadows,arrows.meta}
\tikzset{>={Latex[width=1.5mm,length=2mm]}}
\usepackage{flowchart}\usepackage[paperheight=11.0in,paperwidth=8.5in,left=1.0in,right=1.0in,top=1.0in,bottom=1.0in,headheight=1in]{geometry}
\usepackage[utf8]{inputenc}
\usepackage[T1]{fontenc}
\TabPositions{0.5in,1.0in,1.5in,2.0in,2.5in,3.0in,3.5in,4.0in,4.5in,5.0in,5.5in,6.0in,}

\urlstyle{same}


 %%%%%%%%%%%%  Set Depths for Sections  %%%%%%%%%%%%%%

% 1) Section
% 1.1) SubSection
% 1.1.1) SubSubSection
% 1.1.1.1) Paragraph
% 1.1.1.1.1) Subparagraph


\setcounter{tocdepth}{5}
\setcounter{secnumdepth}{5}


 %%%%%%%%%%%%  Set Depths for Nested Lists created by \begin{enumerate}  %%%%%%%%%%%%%%


\setlistdepth{9}
\renewlist{enumerate}{enumerate}{9}
		\setlist[enumerate,1]{label=\arabic*)}
		\setlist[enumerate,2]{label=\alph*)}
		\setlist[enumerate,3]{label=(\roman*)}
		\setlist[enumerate,4]{label=(\arabic*)}
		\setlist[enumerate,5]{label=(\Alph*)}
		\setlist[enumerate,6]{label=(\Roman*)}
		\setlist[enumerate,7]{label=\arabic*}
		\setlist[enumerate,8]{label=\alph*}
		\setlist[enumerate,9]{label=\roman*}

\renewlist{itemize}{itemize}{9}
		\setlist[itemize]{label=$\cdot$}
		\setlist[itemize,1]{label=\textbullet}
		\setlist[itemize,2]{label=$\circ$}
		\setlist[itemize,3]{label=$\ast$}
		\setlist[itemize,4]{label=$\dagger$}
		\setlist[itemize,5]{label=$\triangleright$}
		\setlist[itemize,6]{label=$\bigstar$}
		\setlist[itemize,7]{label=$\blacklozenge$}
		\setlist[itemize,8]{label=$\prime$}

\setlength{\topsep}{0pt}\setlength{\parskip}{8.04pt}
\setlength{\parindent}{0pt}

 %%%%%%%%%%%%  This sets linespacing (verticle gap between Lines) Default=1 %%%%%%%%%%%%%%


\renewcommand{\arraystretch}{1.3}


%%%%%%%%%%%%%%%%%%%% Document code starts here %%%%%%%%%%%%%%%%%%%%



\begin{document}
\begin{Center}
{\fontsize{15pt}{18.0pt}\selectfont \textbf{SOEN 6481}\par}
\end{Center}\par

\begin{Center}
{\fontsize{15pt}{18.0pt}\selectfont \textbf{Software Systems Requirements Specification}\par}
\end{Center}\par

\begin{Center}
{\fontsize{15pt}{18.0pt}\selectfont \textbf{Summer 2019}\par}
\end{Center}\par

\begin{Center}
{\fontsize{15pt}{18.0pt}\selectfont \textbf{Deliverable 2}\par}
\end{Center}\par

\begin{Center}
{\fontsize{15pt}{18.0pt}\selectfont \textbf{Eternity: Numbers}\par}
\end{Center}\par


\vspace{\baselineskip}
\begin{Center}
{\fontsize{15pt}{18.0pt}\selectfont \textbf{Declaration}\par}
\end{Center}\par


\vspace{\baselineskip}
\begin{justify}
{\fontsize{15pt}{18.0pt}\selectfont \textbf{I have read and understood the Fairness Protocol and Communal Work Protocol, and agree to abide by the policies therein, without any exception under any circumstances, whatsoever.}\par}
\end{justify}\par

\tab \tab 
\vspace{\baselineskip}
\vspace{\baselineskip}

\vspace{\baselineskip}

\vspace{\baselineskip}

\vspace{\baselineskip}
\begin{FlushRight}
{\fontsize{15pt}{18.0pt}\selectfont \textbf{By Bikramjit Singh}\par}
\end{FlushRight}\par

\begin{FlushRight}
{\fontsize{15pt}{18.0pt}\selectfont \textbf{40071229}\par}
\end{FlushRight}\par


\vspace{\baselineskip}

\vspace{\baselineskip}

\vspace{\baselineskip}

\vspace{\baselineskip}

\vspace{\baselineskip}

\vspace{\baselineskip}

\vspace{\baselineskip}

\vspace{\baselineskip}

\vspace{\baselineskip}

\vspace{\baselineskip}

\vspace{\baselineskip}

\vspace{\baselineskip}

\vspace{\baselineskip}

\vspace{\baselineskip}

\vspace{\baselineskip}

\vspace{\baselineskip}

\vspace{\baselineskip}

\vspace{\baselineskip}
\textbf{Table of Contents:}\par

\setlength{\parskip}{0.0pt}
\textbf{Problem\ 6\   ---------------------------------------------------\ \  }Page 3\par

\ \ \ \ \ \ \ \ \ \  \ \ \    \textbf{6.1\ \ \ ---------------------------------------------------\   }Page 3{\fontsize{13pt}{15.6pt}\selectfont \textbf{ }\par}\par

\setlength{\parskip}{2.04pt}
{\fontsize{13pt}{15.6pt}\selectfont \textbf{\ \ \ \ \ \ \ \ \ \ \  6.2\ \ \ ---------------------------------------------------\   }Page 3\par}\par

\textbf{\ \ \ \ \ \ \ \ \ \      \  6.3\ \ \ ---------------------------------------------------\   }Page 4\par

\textbf{\ \ \ \ \ \ \ \ \ \ \  6.4\ \ \ ---------------------------------------------------\   }Page 4\par

\textbf{\ \ \ \ \ \ \ \ \ \ \  6.5\ \ \ ---------------------------------------------------\   }Page 5\par

\textbf{\ \ \ \ \ \ \ \ \ \ \  6.6\ \ \ ---------------------------------------------------\   }Page 5\par

\textbf{\ \ \ \ \ \ \ \ \ \ \  6.7\ \ \ ---------------------------------------------------\   }Page 6\par

\textbf{\ \ \ \ \ \ \ \ \ \ \  6.8\ \ \ ---------------------------------------------------\   }Page 6\par

\textbf{\ \ \ \ \ \ \ \ \ \ \  6.9\ \ \ ---------------------------------------------------\   }Page 7\par

\textbf{\ \ \ \ \ \ \ \ \ \ \ 6.10\ \ \ ---------------------------------------------------   }Page 7\par


\vspace{\baselineskip}
\textbf{ Problem\ 7\   ---------------------------------------------------\   }Page 8\par


\vspace{\baselineskip}
 \textbf{Problem\ 8\ \ \ ---------------------------------------------------   }Page 9\par


\vspace{\baselineskip}
\textbf{ Glossary\  \ \ ----------------------------------------------------\   }Page 10\par


\vspace{\baselineskip}
\textbf{ Repository\ Link\ \ \ --------------------------------------------   }Page 10\par


\vspace{\baselineskip}
\textbf{ References\ \ \ ---------------------------------------------------\   }Page 10\par


\vspace{\baselineskip}

\vspace{\baselineskip}

\vspace{\baselineskip}
 \par


\vspace{\baselineskip}
\setlength{\parskip}{8.04pt}

\vspace{\baselineskip}

\vspace{\baselineskip}

\vspace{\baselineskip}

\vspace{\baselineskip}

\vspace{\baselineskip}

\vspace{\baselineskip}

\vspace{\baselineskip}

\vspace{\baselineskip}

\vspace{\baselineskip}

\vspace{\baselineskip}

\vspace{\baselineskip}

\vspace{\baselineskip}

\vspace{\baselineskip}

\vspace{\baselineskip}

\vspace{\baselineskip}

\vspace{\baselineskip}

\vspace{\baselineskip}

\vspace{\baselineskip}

\vspace{\baselineskip}
{\fontsize{13pt}{15.6pt}\selectfont \textbf{Problem 6 [70 marks]}\par}\par

{\fontsize{13pt}{15.6pt}\selectfont \textbf{USER STORIES}\par}\par

\textbf{6.1 Display input}\par



%%%%%%%%%%%%%%%%%%%% Table No: 1 starts here %%%%%%%%%%%%%%%%%%%%


\begin{table}[H]
 			\centering
\begin{tabular}{p{1.67in}p{4.42in}}
\hline
%row no:1
\multicolumn{1}{|p{1.67in}}{Identifier} & 
\multicolumn{1}{|p{4.42in}|}{EN\_01} \\
\hhline{--}
%row no:2
\multicolumn{1}{|p{1.67in}}{Statement} & 
\multicolumn{1}{|p{4.42in}|}{As a user, I should be able to see the operators/operations on the display screen that I am entering so that I can modify the expression if needed.} \\
\hhline{--}
%row no:3
\multicolumn{1}{|p{1.67in}}{Constraint} & 
\multicolumn{1}{|p{4.42in}|}{The display screen will show a total of 16 inputs of the expression at a time.} \\
\hhline{--}
%row no:4
\multicolumn{1}{|p{1.67in}}{Acceptance Criteria} & 
\multicolumn{1}{|p{4.42in}|}{Given I want to get the sum of 2 $\&$ 3, I should be able to see the expression as 2+3, on the display screen before I click $``$=$"$  for the result.} \\
\hhline{--}
%row no:5
\multicolumn{1}{|p{1.67in}}{Priority} & 
\multicolumn{1}{|p{4.42in}|}{Must have} \\
\hhline{--}
%row no:6
\multicolumn{1}{|p{1.67in}}{Estimate} & 
\multicolumn{1}{|p{4.42in}|}{1} \\
\hhline{--}

\end{tabular}
 \end{table}


%%%%%%%%%%%%%%%%%%%% Table No: 1 ends here %%%%%%%%%%%%%%%%%%%%


\vspace{\baselineskip}
\textbf{6.2 Modify input}\par



%%%%%%%%%%%%%%%%%%%% Table No: 2 starts here %%%%%%%%%%%%%%%%%%%%


\begin{table}[H]
 			\centering
\begin{tabular}{p{1.67in}p{4.42in}}
\hline
%row no:1
\multicolumn{1}{|p{1.67in}}{Identifier} & 
\multicolumn{1}{|p{4.42in}|}{EN\_02} \\
\hhline{--}
%row no:2
\multicolumn{1}{|p{1.67in}}{Statement} & 
\multicolumn{1}{|p{4.42in}|}{As a user, I should be able to delete wrongly entered operators/operations so that I can correct the expression I want to calculate. } \\
\hhline{--}
%row no:3
\multicolumn{1}{|p{1.67in}}{Constraint} & 
\multicolumn{1}{|p{4.42in}|}{The user must have to enter atleast one operator/operation to delete it. \par The user can only delete the last entered input one at a time.} \\
\hhline{--}
%row no:4
\multicolumn{1}{|p{1.67in}}{Acceptance Criteria} & 
\multicolumn{1}{|p{4.42in}|}{1. Given I have entered 1, 2, 3 (123). On pressing the backspace key, I should see 12 on the display screen. \par 2. Given I entered 5. On pressing the backspace key, I should see 0 on the display screen.} \\
\hhline{--}
%row no:5
\multicolumn{1}{|p{1.67in}}{Priority} & 
\multicolumn{1}{|p{4.42in}|}{Must have} \\
\hhline{--}
%row no:6
\multicolumn{1}{|p{1.67in}}{Estimate} & 
\multicolumn{1}{|p{4.42in}|}{1} \\
\hhline{--}

\end{tabular}
 \end{table}


%%%%%%%%%%%%%%%%%%%% Table No: 2 ends here %%%%%%%%%%%%%%%%%%%%


\vspace{\baselineskip}

\vspace{\baselineskip}

\vspace{\baselineskip}

\vspace{\baselineskip}

\vspace{\baselineskip}

\vspace{\baselineskip}

\vspace{\baselineskip}

\vspace{\baselineskip}
\textbf{6.3 Accuracy}\par



%%%%%%%%%%%%%%%%%%%% Table No: 3 starts here %%%%%%%%%%%%%%%%%%%%


\begin{table}[H]
 			\centering
\begin{tabular}{p{1.67in}p{4.42in}}
\hline
%row no:1
\multicolumn{1}{|p{1.67in}}{Identifier} & 
\multicolumn{1}{|p{4.42in}|}{EN\_03} \\
\hhline{--}
%row no:2
\multicolumn{1}{|p{1.67in}}{Statement} & 
\multicolumn{1}{|p{4.42in}|}{As a user I want my calculator to calculate the value of $``$Pi$"$  upto 15 decimal places (enough for any measurable experiment), so that I can use it to bring accuracy in my results} \\
\hhline{--}
%row no:3
\multicolumn{1}{|p{1.67in}}{Constraint} & 
\multicolumn{1}{|p{4.42in}|}{None} \\
\hhline{--}
%row no:4
\multicolumn{1}{|p{1.67in}}{Acceptance Criteria} & 
\multicolumn{1}{|p{4.42in}|}{1. Given I have clicked $``$$ \pi $ $"$  key and then $``$=$"$  , I should see 3.141592653589793 on the display screen} \\
\hhline{--}
%row no:5
\multicolumn{1}{|p{1.67in}}{Priority} & 
\multicolumn{1}{|p{4.42in}|}{Must have} \\
\hhline{--}
%row no:6
\multicolumn{1}{|p{1.67in}}{Estimate} & 
\multicolumn{1}{|p{4.42in}|}{2} \\
\hhline{--}

\end{tabular}
 \end{table}


%%%%%%%%%%%%%%%%%%%% Table No: 3 ends here %%%%%%%%%%%%%%%%%%%%


\vspace{\baselineskip}
\textbf{6.4 Precision}\par



%%%%%%%%%%%%%%%%%%%% Table No: 4 starts here %%%%%%%%%%%%%%%%%%%%


\begin{table}[H]
 			\centering
\begin{tabular}{p{1.67in}p{4.42in}}
\hline
%row no:1
\multicolumn{1}{|p{1.67in}}{Identifier} & 
\multicolumn{1}{|p{4.42in}|}{EN\_04} \\
\hhline{--}
%row no:2
\multicolumn{1}{|p{1.67in}}{Statement} & 
\multicolumn{1}{|p{4.42in}|}{As a user, I should be able to choose the value of $``$pi$"$  upto desired number of decimal points depending on my calculation, to get the precise result.} \\
\hhline{--}
%row no:3
\multicolumn{1}{|p{1.67in}}{Constraint} & 
\multicolumn{1}{|p{4.42in}|}{The minimum decimal places user can choose is 2 and maximum is 15.} \\
\hhline{--}
%row no:4
\multicolumn{1}{|p{1.67in}}{Acceptance Criteria} & 
\multicolumn{1}{|p{4.42in}|}{1. Given I have clicked $``$pi$"$  , on clicking $``$D$"$  key if I enter 2 , the pi in my calculation will have value upto 2 decimal places i.e. 3.14 , which will be enough for me to calculate the circumference of a circle. \par 2. Minimum decimal value to be chosen must be 2 and maximum is 15. \par 3. Even If I click an integer less than 2 or greater than 15, after pressing the $``$D$"$  key, $``$INVALID INPUT$"$  will be displayed on the screen.} \\
\hhline{--}
%row no:5
\multicolumn{1}{|p{1.67in}}{Priority} & 
\multicolumn{1}{|p{4.42in}|}{Should have} \\
\hhline{--}
%row no:6
\multicolumn{1}{|p{1.67in}}{Estimate} & 
\multicolumn{1}{|p{4.42in}|}{2 \par } \\
\hhline{--}

\end{tabular}
 \end{table}


%%%%%%%%%%%%%%%%%%%% Table No: 4 ends here %%%%%%%%%%%%%%%%%%%%


\vspace{\baselineskip}

\vspace{\baselineskip}

\vspace{\baselineskip}

\vspace{\baselineskip}

\vspace{\baselineskip}

\vspace{\baselineskip}

\vspace{\baselineskip}

\vspace{\baselineskip}

\vspace{\baselineskip}

\vspace{\baselineskip}
\textbf{6.5 Area Shortcut}\par



%%%%%%%%%%%%%%%%%%%% Table No: 5 starts here %%%%%%%%%%%%%%%%%%%%


\begin{table}[H]
 			\centering
\begin{tabular}{p{1.67in}p{4.42in}}
\hline
%row no:1
\multicolumn{1}{|p{1.67in}}{Identifier} & 
\multicolumn{1}{|p{4.42in}|}{EN\_05} \\
\hhline{--}
%row no:2
\multicolumn{1}{|p{1.67in}}{Statement} & 
\multicolumn{1}{|p{4.42in}|}{As a user, I should be able to directly calculate the area of a circle by just providing the value of radius as this is the most needed calculation in my field of work thus it will save me a lot of time.} \\
\hhline{--}
%row no:3
\multicolumn{1}{|p{1.67in}}{Constraint} & 
\multicolumn{1}{|p{4.42in}|}{The value of Radius should be greater than zero. (r >0) \par The default value of $``$pi$"$  in this function would be upto 2 decimal places.} \\
\hhline{--}
%row no:4
\multicolumn{1}{|p{1.67in}}{Acceptance Criteria} & 
\multicolumn{1}{|p{4.42in}|}{1. Given I have clicked/typed $``$Area$"$  key. I will now click 5 (radius) to get the result 78.5 (3.14 $\ast$  5 $\ast$  5) displayed on the screen. \par 2. Given I have clicked $``$Area$"$  key. Now on giving the value of radius as -5, the display screen will show $``$INVALID RADIUS$"$ .} \\
\hhline{--}
%row no:5
\multicolumn{1}{|p{1.67in}}{Priority} & 
\multicolumn{1}{|p{4.42in}|}{Should have} \\
\hhline{--}
%row no:6
\multicolumn{1}{|p{1.67in}}{Estimate} & 
\multicolumn{1}{|p{4.42in}|}{2} \\
\hhline{--}

\end{tabular}
 \end{table}


%%%%%%%%%%%%%%%%%%%% Table No: 5 ends here %%%%%%%%%%%%%%%%%%%%


\vspace{\baselineskip}
\textbf{6.6 Store Result}\par



%%%%%%%%%%%%%%%%%%%% Table No: 6 starts here %%%%%%%%%%%%%%%%%%%%


\begin{table}[H]
 			\centering
\begin{tabular}{p{1.67in}p{4.42in}}
\hline
%row no:1
\multicolumn{1}{|p{1.67in}}{Identifier} & 
\multicolumn{1}{|p{4.42in}|}{EN\_06} \\
\hhline{--}
%row no:2
\multicolumn{1}{|p{1.67in}}{Statement} & 
\multicolumn{1}{|p{4.42in}|}{As a user, I should be able to save the results of my calculation so that I can use it later.} \\
\hhline{--}
%row no:3
\multicolumn{1}{|p{1.67in}}{Constraint} & 
\multicolumn{1}{|p{4.42in}|}{There must be atleast one entered value or computation done, after starting the calculator. \par All these saved result will be erased after the user turns off the calculator.} \\
\hhline{--}
%row no:4
\multicolumn{1}{|p{1.67in}}{Acceptance Criteria} & 
\multicolumn{1}{|p{4.42in}|}{1. Given I have calculated area of a circle with radius $``$5$"$  , if I click $``$S$"$  key, the result will be stored in calculator’s memory for later use and a message $``$Result saved$"$  and the memory [5] will be displayed.. \par 2. Given that I have just turned on the calculator and the display screen shows $``$0$"$ , on clicking $``$S$"$  nothing is stored in the memory.} \\
\hhline{--}
%row no:5
\multicolumn{1}{|p{1.67in}}{Priority} & 
\multicolumn{1}{|p{4.42in}|}{Should have} \\
\hhline{--}
%row no:6
\multicolumn{1}{|p{1.67in}}{Estimate} & 
\multicolumn{1}{|p{4.42in}|}{2} \\
\hhline{--}

\end{tabular}
 \end{table}


%%%%%%%%%%%%%%%%%%%% Table No: 6 ends here %%%%%%%%%%%%%%%%%%%%


\vspace{\baselineskip}

\vspace{\baselineskip}

\vspace{\baselineskip}

\vspace{\baselineskip}

\vspace{\baselineskip}

\vspace{\baselineskip}

\vspace{\baselineskip}

\vspace{\baselineskip}

\vspace{\baselineskip}
\textbf{6.7 Clear Memory}\par



%%%%%%%%%%%%%%%%%%%% Table No: 7 starts here %%%%%%%%%%%%%%%%%%%%


\begin{table}[H]
 			\centering
\begin{tabular}{p{1.67in}p{4.42in}}
\hline
%row no:1
\multicolumn{1}{|p{1.67in}}{Identifier} & 
\multicolumn{1}{|p{4.42in}|}{EN\_07} \\
\hhline{--}
%row no:2
\multicolumn{1}{|p{1.67in}}{Statement} & 
\multicolumn{1}{|p{4.42in}|}{As a user, I should be able to delete all the saved results to have a fresh memory and delete the results that will not be needed anymore.} \\
\hhline{--}
%row no:3
\multicolumn{1}{|p{1.67in}}{Constraint} & 
\multicolumn{1}{|p{4.42in}|}{There must be atleast one saved calculation in the memory.} \\
\hhline{--}
%row no:4
\multicolumn{1}{|p{1.67in}}{Acceptance Criteria} & 
\multicolumn{1}{|p{4.42in}|}{1. Given I have saved 5 (3+2), 6 (9-3) in my calculator memory. On clicking $``$MC$"$  key, both the saved values will be deleted. } \\
\hhline{--}
%row no:5
\multicolumn{1}{|p{1.67in}}{Priority} & 
\multicolumn{1}{|p{4.42in}|}{Should have} \\
\hhline{--}
%row no:6
\multicolumn{1}{|p{1.67in}}{Estimate} & 
\multicolumn{1}{|p{4.42in}|}{1} \\
\hhline{--}

\end{tabular}
 \end{table}


%%%%%%%%%%%%%%%%%%%% Table No: 7 ends here %%%%%%%%%%%%%%%%%%%%


\vspace{\baselineskip}
\textbf{6.8 See Stored Results}\par



%%%%%%%%%%%%%%%%%%%% Table No: 8 starts here %%%%%%%%%%%%%%%%%%%%


\begin{table}[H]
 			\centering
\begin{tabular}{p{1.67in}p{4.42in}}
\hline
%row no:1
\multicolumn{1}{|p{1.67in}}{Identifier} & 
\multicolumn{1}{|p{4.42in}|}{EN\_08} \\
\hhline{--}
%row no:2
\multicolumn{1}{|p{1.67in}}{Statement} & 
\multicolumn{1}{|p{4.42in}|}{As a user, I should be able to see all my saved calculations so that I can decide which one to use.} \\
\hhline{--}
%row no:3
\multicolumn{1}{|p{1.67in}}{Constraint} & 
\multicolumn{1}{|p{4.42in}|}{The user must have to save at least one computation in the saving list.} \\
\hhline{--}
%row no:4
\multicolumn{1}{|p{1.67in}}{Acceptance Criteria} & 
\multicolumn{1}{|p{4.42in}|}{1. Given I have saved 2+2=4 and then 4-3=1 by clicking $``$S$"$  key after each computation. On clicking $``$H$"$  key I should see 4, 1. \par 2. Given I have just started the calculator and no value is saved. On clicking $``$H$"$ , $``$No value$"$  will be displayed on the screen} \\
\hhline{--}
%row no:5
\multicolumn{1}{|p{1.67in}}{Priority} & 
\multicolumn{1}{|p{4.42in}|}{Should have} \\
\hhline{--}
%row no:6
\multicolumn{1}{|p{1.67in}}{Estimate} & 
\multicolumn{1}{|p{4.42in}|}{1} \\
\hhline{--}

\end{tabular}
 \end{table}


%%%%%%%%%%%%%%%%%%%% Table No: 8 ends here %%%%%%%%%%%%%%%%%%%%


\vspace{\baselineskip}

\vspace{\baselineskip}

\vspace{\baselineskip}

\vspace{\baselineskip}

\vspace{\baselineskip}

\vspace{\baselineskip}

\vspace{\baselineskip}

\vspace{\baselineskip}

\vspace{\baselineskip}

\vspace{\baselineskip}

\vspace{\baselineskip}

\vspace{\baselineskip}

\vspace{\baselineskip}
\textbf{6.9 Last Calculation}\par



%%%%%%%%%%%%%%%%%%%% Table No: 9 starts here %%%%%%%%%%%%%%%%%%%%


\begin{table}[H]
 			\centering
\begin{tabular}{p{1.67in}p{4.42in}}
\hline
%row no:1
\multicolumn{1}{|p{1.67in}}{Identifier} & 
\multicolumn{1}{|p{4.42in}|}{EN\_09} \\
\hhline{--}
%row no:2
\multicolumn{1}{|p{1.67in}}{Statement} & 
\multicolumn{1}{|p{4.42in}|}{As a user, I should be able to operate on the last calculated answer directly.} \\
\hhline{--}
%row no:3
\multicolumn{1}{|p{1.67in}}{Constraint} & 
\multicolumn{1}{|p{4.42in}|}{The user must have calculated one expression} \\
\hhline{--}
%row no:4
\multicolumn{1}{|p{1.67in}}{Acceptance Criteria} & 
\multicolumn{1}{|p{4.42in}|}{1. Given I calculated 2+2=4, if I click $``$Ans$"$  key and then + 2 , the result displayed should be 6 (4 + 2) \par 2. Given I have not done any calculation, clicking $``$Ans$"$  will do nothing.} \\
\hhline{--}
%row no:5
\multicolumn{1}{|p{1.67in}}{Priority} & 
\multicolumn{1}{|p{4.42in}|}{Should have} \\
\hhline{--}
%row no:6
\multicolumn{1}{|p{1.67in}}{Estimate} & 
\multicolumn{1}{|p{4.42in}|}{1} \\
\hhline{--}

\end{tabular}
 \end{table}


%%%%%%%%%%%%%%%%%%%% Table No: 9 ends here %%%%%%%%%%%%%%%%%%%%


\vspace{\baselineskip}

\vspace{\baselineskip}
\textbf{6.10 iCloud}\par



%%%%%%%%%%%%%%%%%%%% Table No: 10 starts here %%%%%%%%%%%%%%%%%%%%


\begin{table}[H]
 			\centering
\begin{tabular}{p{1.67in}p{4.42in}}
\hline
%row no:1
\multicolumn{1}{|p{1.67in}}{Identifier} & 
\multicolumn{1}{|p{4.42in}|}{EN\_10} \\
\hhline{--}
%row no:2
\multicolumn{1}{|p{1.67in}}{Statement} & 
\multicolumn{1}{|p{4.42in}|}{As a user, I should be able to upload my saved calculations in icloud, so that I can access them from anywhere, anytime.} \\
\hhline{--}
%row no:3
\multicolumn{1}{|p{1.67in}}{Constraint} & 
\multicolumn{1}{|p{4.42in}|}{The user should have an icloud account.} \\
\hhline{--}
%row no:4
\multicolumn{1}{|p{1.67in}}{Acceptance Criteria} & 
\multicolumn{1}{|p{4.42in}|}{ Given I have done a calculation and saved it in the memory, on clicking $``$iC$"$ , the saved result(s) is permanently stored in my icloud account.} \\
\hhline{--}
%row no:5
\multicolumn{1}{|p{1.67in}}{Priority} & 
\multicolumn{1}{|p{4.42in}|}{Could have} \\
\hhline{--}
%row no:6
\multicolumn{1}{|p{1.67in}}{Estimate} & 
\multicolumn{1}{|p{4.42in}|}{3} \\
\hhline{--}

\end{tabular}
 \end{table}


%%%%%%%%%%%%%%%%%%%% Table No: 10 ends here %%%%%%%%%%%%%%%%%%%%


\vspace{\baselineskip}

\vspace{\baselineskip}

\vspace{\baselineskip}

\vspace{\baselineskip}

\vspace{\baselineskip}

\vspace{\baselineskip}

\vspace{\baselineskip}

\vspace{\baselineskip}

\vspace{\baselineskip}

\vspace{\baselineskip}

\vspace{\baselineskip}

\vspace{\baselineskip}

\vspace{\baselineskip}

\vspace{\baselineskip}

\vspace{\baselineskip}

\vspace{\baselineskip}

\vspace{\baselineskip}

\vspace{\baselineskip}

\vspace{\baselineskip}

\vspace{\baselineskip}
{\fontsize{13pt}{15.6pt}\selectfont \textbf{Problem 7 [10 marks]}\par}\par

{\fontsize{13pt}{15.6pt}\selectfont \textbf{Traceability Matrix}\par}\par


\vspace{\baselineskip}


%%%%%%%%%%%%%%%%%%%% Table No: 11 starts here %%%%%%%%%%%%%%%%%%%%


\begin{table}[H]
 			\centering
\begin{tabular}{p{0.36in}p{0.74in}p{1.11in}p{3.48in}}
\hline
%row no:1
\multicolumn{1}{|p{0.36in}}{\textbf{Index}} & 
\multicolumn{1}{|p{0.74in}}{\textbf{User Story Identifier}} & 
\multicolumn{1}{|p{1.11in}}{\Centering \textbf{User Story Name}} & 
\multicolumn{1}{|p{3.48in}|}{\Centering \textbf{User Story Source}} \\
\hhline{----}
%row no:2
\multicolumn{1}{|p{0.36in}}{\Centering 1} & 
\multicolumn{1}{|p{0.74in}}{\Centering EN\_01} & 
\multicolumn{1}{|p{1.11in}}{\Centering Display input} & 
\multicolumn{1}{|p{3.48in}|}{Problem 5 UML Use Case Diagram [ Use Case: Display Result]\href{https://www.online-calculator.com/}{}} \\
\hhline{----}
%row no:3
\multicolumn{1}{|p{0.36in}}{\Centering 2} & 
\multicolumn{1}{|p{0.74in}}{\Centering EN\_02} & 
\multicolumn{1}{|p{1.11in}}{\Centering Modify input} & 
\multicolumn{1}{|p{3.48in}|}{Problem 5 UML Use Case Diagram [ Use Case: Edit Displayed Result] \par } \\
\hhline{----}
%row no:4
\multicolumn{1}{|p{0.36in}}{\Centering 3 \par } & 
\multicolumn{1}{|p{0.74in}}{\Centering EN\_03} & 
\multicolumn{1}{|p{1.11in}}{\Centering Accuracy} & 
\multicolumn{1}{|p{3.48in}|}{Interview (Problem 2)} \\
\hhline{----}
%row no:5
\multicolumn{1}{|p{0.36in}}{\Centering 4} & 
\multicolumn{1}{|p{0.74in}}{\Centering EN\_04} & 
\multicolumn{1}{|p{1.11in}}{\Centering Precision} & 
\multicolumn{1}{|p{3.48in}|}{User Story EN\_03 \par } \\
\hhline{----}
%row no:6
\multicolumn{1}{|p{0.36in}}{\Centering 5 \par } & 
\multicolumn{1}{|p{0.74in}}{\Centering EN\_05} & 
\multicolumn{1}{|p{1.11in}}{\Centering Area Shortcut} & 
\multicolumn{1}{|p{3.48in}|}{Problem 1} \\
\hhline{----}
%row no:7
\multicolumn{1}{|p{0.36in}}{\Centering 6} & 
\multicolumn{1}{|p{0.74in}}{\Centering EN\_06} & 
\multicolumn{1}{|p{1.11in}}{\Centering Store Result} & 
\multicolumn{1}{|p{3.48in}|}{Persona (Problem 3) \par } \\
\hhline{----}
%row no:8
\multicolumn{1}{|p{0.36in}}{\Centering 7 \par } & 
\multicolumn{1}{|p{0.74in}}{\Centering EN\_07} & 
\multicolumn{1}{|p{1.11in}}{\Centering Clear Memory} & 
\multicolumn{1}{|p{3.48in}|}{User Story EN\_06} \\
\hhline{----}
%row no:9
\multicolumn{1}{|p{0.36in}}{\Centering 8} & 
\multicolumn{1}{|p{0.74in}}{\Centering EN\_08} & 
\multicolumn{1}{|p{1.11in}}{\Centering See Stored Results} & 
\multicolumn{1}{|p{3.48in}|}{Interview (Problem 2) \par } \\
\hhline{----}
%row no:10
\multicolumn{1}{|p{0.36in}}{\Centering 9} & 
\multicolumn{1}{|p{0.74in}}{\Centering EN\_09} & 
\multicolumn{1}{|p{1.11in}}{\Centering Last Calculation} & 
\multicolumn{1}{|p{3.48in}|}{\href{https://www.online-calculator.com/}{https://www.online-calculator.com/} \par } \\
\hhline{----}
%row no:11
\multicolumn{1}{|p{0.36in}}{\Centering 10} & 
\multicolumn{1}{|p{0.74in}}{\Centering EN\_10} & 
\multicolumn{1}{|p{1.11in}}{\Centering iCloud} & 
\multicolumn{1}{|p{3.48in}|}{Interview (Problem 2) \par } \\
\hhline{----}

\end{tabular}
 \end{table}


%%%%%%%%%%%%%%%%%%%% Table No: 11 ends here %%%%%%%%%%%%%%%%%%%%


\vspace{\baselineskip}

\vspace{\baselineskip}

\vspace{\baselineskip}

\vspace{\baselineskip}

\vspace{\baselineskip}

\vspace{\baselineskip}

\vspace{\baselineskip}

\vspace{\baselineskip}

\vspace{\baselineskip}

\vspace{\baselineskip}

\vspace{\baselineskip}

\vspace{\baselineskip}

\vspace{\baselineskip}
{\fontsize{13pt}{15.6pt}\selectfont \textbf{Problem 8 [40 marks]}\par}\par

I have implemented \textbf{four user stories} using \textbf{three classes}, whose description is as follow:\par

All the inputs while running these programs are \textbf{NOT CASE SENSITIVE }\par

\textbf{1. EN\_03\_04: }This class shows the implementation of \textbf{User Story 3 (EN\_03) $\&$  User Story 4 (EN\_04)}. The goal of this implementation is to calculate or use the value of pi by letting the user decide till how many decimal places he wants to use the value of pi. If the user chooses/types $``$pi$"$  as the first operand, he is asked if he want to decide the precision of the value of pi. The user can type $``$D$"$  continued with an integer ‘i’ (1<i<16) to choose the number of decimal places he wants the value of pi to be upto or $``$c$"$  to continue with default 15 decimal places. After choosing he can just type $``$=$"$  to calculate the value of pi or choose any arithmetic operator $``$+,-, /,$\ast$ $"$  to operate on first operand $``$pi$"$ . In this program, after end of one operation, user can type $``$c$"$  to continue with another operation or type $``$q$"$  to terminate the program.\par

\textbf{2 EN\_05: }This class shows the implementation of \textbf{User Story 5 (EN\_05)} and shows the application of number $``$pi$"$ in real life. According to the description of number pi in Problem 1 and the interview it is quite clear that pi’s most common use is in finding area or circumference and area of a circle. So in this implementation, user has the option to quickly calculate the area. User will just press $``$Area$"$  button (type $``$Area$"$  while running this program) and then give the value of radius to calculate the area of the circle. After completing one operation, user can press $``$c$"$  to continue or $``$q$"$  to terminate the program.\par

\textbf{3. EN\_06: }This class implements \textbf{User Story 6 (EN\_06) }which focuses on the application of the calculator. User can store the results he wants and use them later. In the program, user enters first operand as a rational number or pi, in this implementation choosing the precision of pi is not implemented as the focus is only to implement $``$saving the result$"$ , followed by one of the four primary arithmetic operators and then the second operator. On getting the result, all user has to do is press/type $``$S$"$  to save the result in calculator’s memory. If he doesn’t want to store this result all he can type $``$c$"$  to continue or $``$q$"$  to quit.\par

 \par


\vspace{\baselineskip}
\setlength{\parskip}{0.0pt}

\vspace{\baselineskip}

\vspace{\baselineskip}
\setlength{\parskip}{8.04pt}

\vspace{\baselineskip}

\vspace{\baselineskip}

\vspace{\baselineskip}

\vspace{\baselineskip}

\vspace{\baselineskip}

\vspace{\baselineskip}

\vspace{\baselineskip}

\vspace{\baselineskip}

\vspace{\baselineskip}

\vspace{\baselineskip}

\vspace{\baselineskip}
{\fontsize{13pt}{15.6pt}\selectfont \textbf{Glossary:}\par}\par

\setlength{\parskip}{6.0pt}
1. User Story:\textbf{ }In \href{https://en.wikipedia.org/wiki/Software_development}{software development} and \href{https://en.wikipedia.org/wiki/Product_management}{product management}, a user story is an informal, natural language description of one or more features of a software system, often written from the perspective of an \href{https://en.wikipedia.org/wiki/User_(computing)}{end user} or \href{https://en.wikipedia.org/wiki/User_(system)}{user of a system}\textbf{.}\par

\setlength{\parskip}{2.04pt}
2. Traceability Matrix: In \href{https://en.wikipedia.org/wiki/Software_development}{software development}, a traceability matrix is a document, usually in the form of a table, used to assist in determining the completeness of a relationship by correlating any two baselined documents using a many-to-many relationship comparison.\par


\vspace{\baselineskip}
{\fontsize{13pt}{15.6pt}\selectfont \textbf{Repository Link: }\par}\par


\vspace{\baselineskip}
\href{https://github.com/Bikram1907/SOEN-6481}{\textcolor[HTML]{023160}{https://github.com/Bikram1907/SOEN-6481}}\par


\vspace{\baselineskip}
{\fontsize{13pt}{15.6pt}\selectfont \textbf{References: }\par}\par

1. P. Kamthan, summer 2019, $``$Project Description$"$ , Department of Csc. $\&$  SE, Concordia University\par

2. P. Kamthan, summer 2019, $``$User Stories in Context$"$ , Department of Csc. $\&$  SE, Concordia University.\par

3. P. Kamthan, summer 2019, $``$Traceability in Software Requirements$"$ , Department of Csc. $\&$  SE, Concordia University.\par

4. $``$Four Function Calculator Requirement Specification$"$ \par

\href{http://www.mathcs.richmond.edu/~barnett/cs322/assignments/1999_fall/calculator_requirements.pdf}{http://www.mathcs.richmond.edu/$ \sim $ barnett/cs322/assignments/1999\_fall/calculator\_requirements.pdf}\par

5. $``$Software Requirements Document: A Multi-Function Calculator$"$  \href{http://www2.cs.uidaho.edu/~rinker/cs113/calculator.pdf}{http://www2.cs.uidaho.edu/$ \sim $ rinker/cs113/calculator.pdf}\par


\vspace{\baselineskip}

\vspace{\baselineskip}

\printbibliography
\end{document}